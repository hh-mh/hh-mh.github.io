%\documentclass[slidestop,blue,9pt]{beamer}
\documentclass[slidestop,blue,handout,9pt]{beamer}
\usepackage[utf8]{inputenc} 
\newcommand{\myuncover}{invisible}
\newcommand{\mytitle}{MA2047 Algebra och diskret matematik } 
\newcommand{\mysubtitle}{Något om komplexa tal} 
\newcommand{\mypagesubtitle}{Något om komplexa tal}
\newcommand{\mynames}{Mikael Hindgren} 
\usepackage{polynom}
%\input{longdiv}
\input{/home/mikael/texmf/preamble_hh_lecture_sv}
\begin{document} 

%-------------------------------------------------------------------------------
\begin{frame}[plain]
%\beamertemplatenavigationsymbolsempty
%  \setbeamertemplate{navigation symbols}{}
  \titlepage
\end{frame}
%-------------------------------------------------------------------------------
%\begin{frame}
%  \frametitle{Innehåll}
%  \tableofcontents
%  % You might wish to add the option [pausesections]
%\end{frame}
%###############################################################################

\section{Det komplexa talområdet} 
%===============================================================================
\subsection{Komplexa tal på rektangulär form}

%-------------------------------------------------------------------------------
\begin{frame}
\frametitle{Den imaginära enheten $i$}
\pause
%\onslide<+->
Det finns inga \emph{reella} tal som uppfyller ekvationen $x^{2}+1=0$.
\onslide<+->
\begin{center}
\begin{minipage}{0.78\textwidth}
\begin{varblock}[0.8\textwidth]{}
Vi inför den \alert{imaginära enheten $i$} med egenskapen 
\[ 
i^2 = -1 
\]
\end{varblock}
\end{minipage}
\end{center}
\onslide<+->
Ekvationen $x^ 2 + 1 = 0$ har då lösningen $x^2 = -1 = i^2 \Leftrightarrow x = \pm i$\\
% \begin{itemize}
% \item Det finns inga \emph{reella} tal som uppfyller ekvationen
%  $x^{2}+1=0$.
% \item För att ekvationen ska få lösning inför vi
% den \alert{imaginära enheten $i$} med egenskapen 
% \[
% i^2 = -1
% \]
% \item Ekvationen $x^ 2 + 1 = 0$ har då lösningen 
% \[
% x^2 = -1 = i^2 \Leftrightarrow x = \pm i
% \]
% \end{itemize}
%\onslide<+->
\begin{exempel}
Lös ekvationen $x^2 + 4 = 0$.
\onslide<+->
\begin{losning} 
\[
x^2 = -4 = \onslide<+-> (-1)\cdot 4 = \onslide<+-> i^2\cdot 4 
\onslide<+-> \Leftrightarrow x = \pm 2i
\]
\end{losning}
\end{exempel}
\end{frame}
 
%-------------------------------------------------------------------------------
\begin{frame}
\frametitle{Det komplexa talområdet}
\begin{exempel}
Lös ekvationen $x^2 + 2x + 10 = 0$.
\onslide<+->
\begin{losning}
\vspace{-0.5cm}
\begin{eqnarray*}
x^2 + 2x + 10 &=& \onslide<+-> (x+1)^2 - 1 + 10 
\onslide<+->
= (x+1)^2 + 9 = 0 \\
\onslide<+->
\Leftrightarrow (x+1)^2 &=& -9 = \onslide<+-> (-1)9 = \onslide<+-> i^29 \\
\onslide<+->
\Leftrightarrow x + 1 &=&
\pm 3i \\
\onslide<+->
\Leftrightarrow x &=& {\color{hhblue1}-1} {\color{hhred} \pm \,3}i
\end{eqnarray*}
\end{losning}
\end{exempel}
\onslide<+->
Lösningarna består av en reell del ({\color{hhblue1}$-1$}) och en
imaginär del ({\color{hhred}$3$} respektive {\color{hhred}$-3$}). 
\begin{anm}
$pq$-formeln om $(\frac{p}{2})^2 < q$:
\begin{eqnarray*}
x^2+px+q \mkern-10mu &=& \mkern-10mu \onslide<+-> \textstyle (x+\frac{p}{2})^2 -(\frac{p}{2})^2+q = 0\\
\onslide<+->
\Leftrightarrow \textstyle (x+\frac{p}{2})^2 
\mkern-10mu &= &\mkern-10mu \underbrace{\textstyle (\frac{p}{2})^2-q}_{<0}
\onslide<+-> 
= \underbrace{i^2}_{=-1}\underbrace{\textstyle (q-(\frac{p}{2})^2)}_{>0}\\
\onslide<+->
\Leftrightarrow 
x \mkern-10mu &=& \mkern-10mu \textstyle-\frac{p}{2} \pm i\sqrt{q-(\frac{p}{2})^2}
\end{eqnarray*}
% \begin{enumerate}
% \item $(\frac{p}{2})^2 \geq q$: $ x+\frac{p}{q} = \pm \sqrt{(\frac{p}{2})^2-q}
% \Leftrightarrow x = -\frac{p}{q} \pm \sqrt{(\frac{p}{2})^2-q}$
% \item $(\frac{p}{2})^2 < q$: $ x+\frac{p}{q} = \pm \sqrt{(\frac{p}{2})^2-q}
% \Leftrightarrow x = -\frac{p}{q} \pm \sqrt{(\frac{p}{2})^2-q}$
% \end{enumerate}
% \[
%  = -1 \pm \sqrt{1-10} = -1 \pm i\sqrt{10-1} = -1 \pm 3i
% \]  
\end{anm}
\end{frame}

%-------------------------------------------------------------------------------
\begin{frame}
\frametitle{Det komplexa talområdet}
\begin{definition}[Komplexa talområdet]
Mängden av tal $z=a+ib$, där $a,b\in \R$, kallas 
\alert{det komplexa talområdet $\mathbb{C}$}. 
\begin{itemize}
\item $a = $ \alert{realdelen} av $z$ ($\re{z}$)
\item $b = $ \alert{imaginärdelen} av $z$ ($\im{z}$)
\end{itemize}
\onslide<4->Om $\re{z} = 0$ är $z$ \alert{imaginärt}.
\end{definition}
\onslide<5->
\begin{anm}
\begin{itemize}
 \item Imaginärdelen av ett komplext tal är ett \emph{reellt tal}: \\
\onslide<+->
Ex: $z = 2 - 3i\Rightarrow \im{z} = -3$
\onslide<6->
\item De reella talen är de komplexa tal vars imaginärdel är noll \\
\onslide<+->
$\Rightarrow$
 $\R$ är en äkta delmängd av $\mathbb{C}$:
\end{itemize}
  \[
  \mathbb{N}\subset \mathbb{Z}\subset \mathbb{Q}\subset
  \mathbb{R}\subset \mathbb{C}
  \]
\end{anm}
\end{frame}

%-------------------------------------------------------------------------------
\begin{frame}
\frametitle{Räkneregler för komplexa tal}

\begin{definition}[Räkneregler]
Om $z_1 = a + ib$ och $z_2 = c + id$ är två komplexa tal och $x$ ett
reellt tal så definierar vi \emph{likhet}, \emph{addition, subtraktion}
och \emph{multiplikation} enligt:

\begin{enumerate}
\item  $z_1 = z_2 \Leftrightarrow a = c$ och $b = d$

\item  $xz_1 = xa + ixb$

\item  $z_1 + z_2 = (a + ib) + (c + id) = a + c + i(b + d)$

\item  $z_1 - z_2 = z_1 + (-1)z_2$

\item  $z_1\cdot z_2 = (a+ib)(c+id) = ac + iad + ibc + i^2bd 
                     = ac - bd + i(ad + bc)$
\end{enumerate}
\end{definition}
\begin{sats}
Lagarna för addition, multiplikation och subtraktion av reella tal gäller 
också för komplexa tal.
\end{sats}
%\onslide<+->
\begin{anm}
Vi kan alltså räkna med komplexa tal precis som med reella om vi
tar hänsyn till att $i^2 = -1.$
\end{anm}
\end{frame}


%-------------------------------------------------------------------------------
\begin{frame}
\frametitle{Räkneregler för komplexa tal}
\begin{exempel}
Bestäm $z_1 + z_2$, $z_1\cdot z_2$ och $z_1 - z_2$ om $z_1 = 2 + 3i$ och $z_2 = 5 - 4i$.
 \onslide<+->
\begin{losning}
\begin{eqnarray*}
z_1 + z_2 & =& \onslide<+->  2 + 3i + 5 - 4i = \onslide<+->7-i \\
\onslide<+->
z_1\cdot z_2    & =& \onslide<+-> (2 + 3i)(5 - 4i) 
= \onslide<+-> 10 - 8i + 15i - 12i^2 = \onslide<+-> 22 + 7i \\
\onslide<+->
z_1 - z_2 & =& \onslide<+->z_1 + (-1)z_2 = \onslide<+->2 + 3i + (-1)(5 - 4i) = \onslide<+->-3+7i
\end{eqnarray*}
\end{losning}
\end{exempel}
\end{frame}

%-------------------------------------------------------------------------------
\begin{frame} 
\frametitle{Komplexa tal och olikheter}
Kan vi definiera olikheter för komplexa tal som uppfyller de
vanliga lagarna för olikheter mellan reella tal? \\
\pause
För $a,b,c\in\R$ har vi t ex
\[
c > 0  \text{ och } a < b \Rightarrow ac < bc \qquad (\text{Ex:}\quad 2<3\Rightarrow 2\cdot 4<3\cdot 4)
\]
\pause
Vi väljer talen $0$ och $i$:
\onslide<+->
\begin{itemize}
\item $i \neq 0 \Rightarrow i>0$ eller $i<0$
\item Antag att $i> 0$:
\onslide<+->
\[
\Rightarrow 0 = \underset{a}{0}\cdot \underset{c}{i} < \onslide<+->
\underset{b}{i}\cdot\underset{c}{i}= \onslide<+-> i^2 = -1 \text{ orimligt!}
\]
\item Antag istället att $i < 0 \Leftrightarrow -i > 0$:
\onslide<+->
\[
\Rightarrow 0 = \underset{a}{0}\cdot \underset{c}{(-i)} <\onslide<+->
\underset{b}{(-i)}\cdot\underset{c}{(-i)}= i^2 = \onslide<+->-1 \text{ orimligt!}
\]
\end{itemize}
\begin{block}{}
\alert{Slutsats:} Det går inte att definiera en ordningsrelation på $\mathbb{C}$ som uppfyller de
vanliga ordningslagarna på $\R$. Uttryck av typen $z_1 < z_2$
har ingen mening om vi med ''$<$'' menar den vanliga ordningsrelationen på $\R$.
\end{block} 
% $\therefore$ Det är inte möjligt att definiera $<$ för komplexa tal så att
% ordningsrelationen får samma egenskaper som för reella tal. För

% Uttryck av typen $z_1 < z_2$ har alltså ingen mening om vi med $<$ menar den
% vanliga ordningsrelationen på $R$.
% Det är omöjligt att definiera $<$ för komplexa tal så att
% ordningsrelationen får samma egenskaper som för reella tal. För
% att illustrera detta faktum så kan vi t ex studera talet $i$. Eftersom $i \neq 0$ 
% så måste antingen $i > 0$ eller $i < 0$. Antag först att $i > 0$. Då måste 
% $i\cdot i = i^2 = -1 > 0$ vilket är orimligt och antagandet att $i > 0$ måste vara 
% felaktigt. På motsvarande sätt leder också antagandet att $i < 0$ till en 
% orimlighet..
\end{frame}

\subsection{Det komplexa talplanet}
%-------------------------------------------------------------------------------
\begin{frame}
\frametitle{Det komplexa talplanet}

\begin{itemize}
 \item Ett komplext tal $z=a+ib$ kan tolkas geometriskt som en punkt
  $(a,b)$ eller en vektor i \alert{det komplexa talplanet}
\item $x$-axeln kallas \alert{den reella axeln} och $y$-axeln \alert{den
   imaginära axeln}
\item Addition av två komplexa tal $z_1$ och $z_2$ motsvaras av vektoraddition
\end{itemize}
\vspace{0.3cm}

\begin{minipage}{\textwidth}
\only<1->{\includegraphics[height=2.7cm]{bilder/komplex1.pdf}} 
\only<3->{\includegraphics[height=2.7cm]{bilder/komplexplus.pdf}}
\only<3->{\includegraphics[height=2.7cm]{bilder/komplexminus.pdf}}
\end{minipage}

\vspace{0.2cm}

\only<3->{\small{\hspace{5cm} Addition \hspace{2.5cm} Subtraktion}}
\end{frame}

%-------------------------------------------------------------------------------
\begin{frame}
\frametitle{Absolutbelopp och konjugat}
\begin{itemize}
\item Avståndet mellan talet (punkten) $z=a+ib$ och origo är 
$\sqrt{a^2 + b^2}$
\item Spegelbilden av talet $z=a+ib$ i den reella axeln är talet $a - ib$ 
\end{itemize}
%\begin{center}
%\includegraphics[width=\textwidth]{bilder/komplex2.pdf}
\only<1->{\includegraphics[height=2.5cm]{bilder/komplexabs1.pdf}}
\hspace{0.1cm}
\only<2->{\includegraphics[height=2.5cm]{bilder/komplexkonj.pdf}}
\hspace{0.1cm}
\only<6->{\includegraphics[height=2.5cm]{bilder/komplexabs2.pdf}}
%\end{center}
\begin{definition}
\label{abs}
Om $z = a + ib$, $\,a,b\in \R$, kallas

\begin{itemize}
\item  $|z| = \sqrt{a^2 + b^2}$ \alert{absolutbeloppet} av $z$

\item  $\bar{z} = a - ib$ \alert{komplexkonjugatet} till $z$
\end{itemize}
\end{definition}

\begin{itemize}
\item<6->$|z_1 - z_2|$ är avståndet mellan punkterna $z_1$ och $z_2$
\item<7-> Om $z = x$ där $x\in\R$ är
\[
|z| = \sqrt{x^2+0^2} = \sqrt{x^2} = 
\begin{cases}
x,& x\geq 0\\
-x,& x<0  
\end{cases}
= |x|
\]


\end{itemize}


\end{frame}

%-------------------------------------------------------------------------------
\begin{frame}
\frametitle{Absolutbelopp och konjugat}
\begin{exempel}
Bestäm $z\cdot \bar{z}$, $|\bar{z}|$, $z+\bar{z}$ och $z-\bar{z}$.  
\onslide<+->
\begin{losning}
\begin{eqnarray*}
z\cdot\bar{z}   & =& (a + ib)(a - ib) = \onslide<+->
a^2 - iab + iba - i^2b^2 = \onslide<+->
a^2 + b^2 =\onslide<+->|z|^2 \\
\onslide<+->
|\bar{z}|  & = &\sqrt{a^2 + (-b)^2} = \onslide<+->
\sqrt{a^2 + b^2} =\onslide<+->
|z| \\
\onslide<+->
z + \bar{z}& = &a + ib + a - ib = \onslide<+->
2a = \onslide<+-> 2\re{z} \\
\onslide<+->
z - \bar{z}& = &a + ib - (a - ib) =\onslide<+->2ib = \onslide<+->2i\im{z}
\end{eqnarray*}
\end{losning}
\end{exempel}
\begin{exempel}
\begin{minipage}[t]{0.7\linewidth}
Rita mängden av de komplexa tal $z$ för vilka \\
$|z - 1| < 2$
och $\re{z} \geq 1.$
\end{minipage}
\begin{minipage}[t]{0.25\linewidth}
\vspace{-0.3cm}
\onslide<+->
\only<1->{\includegraphics[height=2.2cm]{bilder/komplexabs3.pdf}}    
\end{minipage}
\end{exempel}
\end{frame}

%-------------------------------------------------------------------------------
\begin{frame}
\frametitle{Absolutbelopp och konjugat}
\begin{exempel}
Lös ekvationen $2z + i\bar{z} = 4 - i$.
\onslide<+->
\begin{losning}
Sätt $z=a + ib$
\begin{eqnarray*}
\onslide<+->
\Rightarrow 2z + i\bar{z} &=& \onslide<+->
2(a + ib) + i(a - ib) = \onslide<+->
2a + b + i(2b + a) \onslide<+->
= 4 - i\\
\onslide<+->
&\Leftrightarrow& 
\begin{cases}
2a + b &=4 \\ 
2b + a &=-1
\end{cases}
\onslide<+->
\Leftrightarrow  
\begin{cases}
a &= 3 \\ 
b &= -2
\end{cases}
\onslide<+->
\Rightarrow z = 3 -2i
\end{eqnarray*}
\end{losning}
\end{exempel}
\end{frame}

%\subsection{Division}
%-------------------------------------------------------------------------------
\begin{frame}
\frametitle{Division}


% När vi multiplicerar ett komplext tal med dess konjugat får vi ett
% reellt tal (absolutbeloppet i kvadrat). 

\begin{exempel}
Hur ska vi definiera kvoten $\displaystyle\frac{5 + 15i}{1 - 3i}$\,?

\onslide<2->
\begin{losning}
\noindent Om vi antar att vi kan räkna på som vanligt:
%så kan vi multiplicera täljaren och nämnaren med konjugatet till nämnaren: 
\onslide<3->
% \begin{equation*}
% \pause
% \frac{2 + 3i}{4 + 5i} 
% \pause = \frac{(2 + 3i)(4 - 5i)}{(4 + 5i)(4 - 5i)}
% \pause = \frac{23 + 2i}{4^2 + 5^2} 
% \pause = \frac{23 + 2i}{41} 
% \pause = \frac{23}{41} + \frac{2}{41}i
% \end{equation*}
\begin{equation*}
\pause
\frac{5 + 15i}{1 - 3i} 
\pause = \frac{(5 +15i)(1 + 3i)}{(1 - 3i)(1 + 3i)}
\pause = \frac{-40 + 30i}{(-1)^2 + 3^2} 
\pause = \frac{-40 + 30i}{10} 
\pause = -4 + 3i
\end{equation*}
\end{losning}
\end{exempel} 

\onslide<8->
\begin{definition}[Division]
Om $z_1$ och $z_2 \neq 0$ är två komplexa tal så definierar vi \emph{kvoten} mellan 
$z_1$ och $z_2$ enligt 
\begin{equation*}
\frac{z_1}{z_2} = \frac{z_1\overline{z}_2}{|z_2|^2}
\end{equation*}
\end{definition}
\end{frame}


% %-------------------------------------------------------------------------------
% \begin{frame}
% \frametitle{Division}

% \alert{Anm:}
% Vi vill självklart att kvoten $\frac{z}{w}$, där $w \neq 0$, skall vara den 
% entydiga lösningen till ekvationen $wx = z$. Att denna ekvation endast har en 
% lösning är uppenbart ty antag att $x_1$ och $x_2$ är två olika lösningar. Då är 
% $wx_1 = z$ och $wx_2 = z$ vilket innebär att $w(x_1 - x_2) = 0$ och eftersom 
% $w \neq 0$ så måste det gälla att $x_1 = x_2$. Att verkligen $x = \frac{z}{w}$
% är en lösning ser vi från 
% \begin{equation*}
% wx = w\frac{z}{w} = w\frac{z\overline{w}}{|w|^2} 
%    = \frac{zw\overline{w}}{|w|^2} = \frac{z|w|^2}{|w|^2} = z.
% \end{equation*}

% Den allmänna regeln vid division är alltså att \emph{multiplicera täljare och 
% nämnare med nämnarens konjugat}.

% \begin{exempel}
% Visa att

% \begin{enumerate}
% \item  $\overline{z_1 + z_2} = \bar{z}_1 + \bar{z}_2$.

% \item  $\overline{z_1 \cdot z_2} = \bar{z}_1 \cdot \bar{z}_2$.

% \item  $z = \bar{z} \Rightarrow z$ reellt.

% \item  $\displaystyle \overline{(\frac{z_1}{z_2})} = \frac{\bar{z}_1}{\bar{z}_2}$.
% Ledning: $\displaystyle z_1 = \frac{z_1}{z_2} \cdot z_2$.
% \end{enumerate}
% \end{exempel}
% \end{frame}

% %-------------------------------------------------------------------------------
% \begin{frame}
% \frametitle{Det komplexa talplanet}

% \begin{exempel}
% Visa att
% \begin{enumerate}
% \item  $|z_1 \cdot z_2| = |z_1| \cdot |z_2|$.

% \item  $\displaystyle \left|\frac{z_1}{z_2}\right| =\frac{|z_1|}{|z_2|},:z_2 \neq 0$.
% \end{enumerate}
% \end{exempel}


% \end{frame}

%-------------------------------------------------------------------------------
\begin{frame}
\frametitle{Triangelolikheten}

Från den geometriska tolkningen av komplexa tal får vi:
\begin{center}
\includegraphics[width=3.5cm]{bilder/komplextriangel.pdf}  
\end{center}
\onslide<2->
\begin{sats}[Triangelolikheten]
För alla komplexa tal $z_1$ och $z_2$ gäller 
\begin{equation*}
|z_1 + z_2| \leq |z_1| + |z_2|
\end{equation*}
% \noindent Vi bör observera att Fig. 4 inte är nå got bevis för
% triangelolikheten, utan bara ett sätt att troliggöra den. Ett
% riktigt bevis må ste göras algebraiskt.
\end{sats}

%\begin{bevis}
% \begin{align*}
% |z_1 + z_2|^2& = (z_1 + z_2)(\overline{z_1 + z_2}) = (z_1 + z_2)(\bar{z}_1 + \bar{z}_2) \\
% & = z_1\bar{z}_1 + z_1\bar{z}_2 + \bar{z}_1z_2 + z_2\bar{z}_2
%   =|z_1|^2 + z_1\bar{z}_2 + \overline{z_1\bar{z}_2} + |z_2|^2 \\
% & =|z_1|^2 + 2\re{(z_{1}\bar{z}_{2})} + |z_2|^2 \leq |z_1|^2 + 2|z_1\bar{z}_2| + |z_2|^2 \\
% & =|z_1|^2 + 2|z_1||z_2| + |z_2|^2 = (|z_1| + |z_2|)^2 \\
% \Leftrightarrow |z_1 + z_2|& \leq |z_1| + |z_2|.
% \end{align*}
%\end{bevis}

% \begin{exempel}
% I beviset av triangelolikheten användes $\re{z} \leq |z|$. Visa denna
% olikhet. 
% \end{exempel}
\onslide<3->
\begin{exempel}
Visa att om $|z| = 1$ så är $|z + 3 + 4i| \leq 6$. Rita figur!
\end{exempel}
\end{frame}

\subsection{Komplexa tal på polär form}
%-------------------------------------------------------------------------------
\begin{frame}
\frametitle{Komplexa tal på polär form}
Den geometriska tolkningen ger oss ett alternativt sätt att representera
ett komplext tal $z$:
\begin{center}
\begin{minipage}{0.4\textwidth}
\includegraphics[width=4.5cm]{bilder/komplexpol1.pdf}
\end{minipage}
\begin{minipage}{0.4\textwidth}
\begin{eqnarray*}
\onslide<2->
\Rightarrow \quad
\begin{cases}
a &= r\cos \theta \\ 
b &= r\sin \theta
\end{cases}
\end{eqnarray*}
\end{minipage}
\end{center}
\onslide<3->
\begin{center}
\begin{minipage}{0.44\linewidth}
\begin{block}{}
$z = \underset{\text{Rektangulär form}}{a + i b} 
= \underset{\text{Polär form}}{r(\cos \theta + i\sin \theta)}$
\end{block}  
\end{minipage}
\end{center}
% Den geometriska tolkningen ger oss ett alternativt sätt att representera
% ett komplext tal $z$. 
% Vi ser (Fig. 5) att $z$ karaktäriseras av längden 
% $r = |z|$ på vektorn $z$ samt av vinkeln $\theta $ mellan den reella axeln och 
% vektorn $z$. Vinkeln $\theta $ räknas som positiv om den motsvaras av en vridning 
% moturs från den reella axeln. Vi observerar också att $\theta $ är bara bestämd på 
% en multipel av $2\pi$ när, ty $z$ beskrivs lika bra av vinkeln $\theta + n\cdot2\pi$ där 
% $n$ är ett godtyckligt heltal.
% \includegraphics[width=3cm]{bilder/komplexpol1.pdf}
% \noindent Från Fig. 5 ser vi att 
\onslide<4->
\begin{itemize}
% \item Talet $z$ säges vara framställt på \emph{polär form}. 
\item $\theta$ kallas \alert{argumentet för $z$} (arg $z$) och räknas positiv om den motsvaras av en vridning moturs från den reella axeln. 
\onslide<5->
\item $\theta$ är inte entydig: 
$r(\cos \theta + i \sin \theta) = r(\cos (\theta + n\cdot 2\pi) + i
\sin (\theta + n\cdot 2\pi))$.
%$z$ beskrivs lika bra av vinkeln $\theta + n\cdot2\pi$ där $n$ är ett godtyckligt heltal.
\onslide<6->
\item Argumentet $\theta$ för vilket $-\pi < \theta \leq \pi $ kallas
  \alert{principalargumentet}.
\end{itemize}
\end{frame}

%-------------------------------------------------------------------------------
\begin{frame}
\frametitle{Komplexa tal på polär form}
\begin{exempel}
Skriv talet $2 - 2i$ på polär form.
\onslide<+->
\begin{textblock*}{31mm}(88mm,0.20\textheight)
\begin{block}{}
\includegraphics[width=3cm]{bilder/komplexpol1.pdf}\\
\footnotesize
\hspace{0.5cm}
$\dst \cos \theta = \frac{a}{r}, \, 
\sin \theta = \frac{b}{r}$\\
\vspace{0.1cm}
\hspace{0.5cm}
$r = |z| = \sqrt{a^2 + b^2}$\\
\vspace{0.1cm}
\hspace{0.5cm}
$z=r(\cos \theta + i \sin \theta)$
\end{block}
\end{textblock*}    

\begin{minipage}[t]{0.6\linewidth}
\begin{losning}
$z = 2 - 2i\Rightarrow$
\onslide<+->
\begin{eqnarray*}
&&r = |z| = \onslide<+-> \sqrt{2^2 + (-2)^2} = \onslide<+->\sqrt{8} = \onslide<+->2\sqrt{2} \\
\onslide<+->
&&\begin{cases}
\cos \theta &= \onslide<+-> \frac{2}{r} = \onslide<+-> \frac{1}{\sqrt{2}} \\
\onslide<+->
\sin \theta &= \onslide<+-> \frac{-2}{r} = \onslide<+-> -\frac{1}{\sqrt{2}}  
\end{cases}
\onslide<+->
\Rightarrow \text{ vi kan välja } \theta = \onslide<+-> \frac{7\pi}{4} \\
\onslide<+->
&\Rightarrow& 2 - 2i = \onslide<+-> 2\sqrt{2}(\cos\frac{7\pi}{4} + i\sin\frac{7\pi}{4})
\end{eqnarray*}
\end{losning}
\end{minipage}
\end{exempel} 
\begin{anm}
Principalargumentet i exemplet ovan är $\dst -\frac{\pi}{4}$.  
\end{anm}
\end{frame}

%\subsection{Den komplexa exponentialfunktionen}
%-------------------------------------------------------------------------------
\begin{frame}
\frametitle{Den komplexa exponentialfunktionen}
% Vi skall nu definiera funktionen $e^z$ där $z$ är komplext tal. Självklart vill 
% vi att definitionen skall vara så dan att de karaktäristiska egenskaperna hos 
% $e^z$ skall vara desamma som för $e^x$ där $x$ är ett reellt tal. För $e^x$ har vi

% \begin{enumerate}
% \item  $e^{x_1} e^{x_2} = e^{x_1 + x_2}$
% \item  $e^0 = 1$
% \item  $D(e^{kx}) = ke^{kx}$
% \end{enumerate}

% Om $z = x + iy$ så vill vi alltså ha $e^z = e^{x + iy} = e^xe^{iy}$. Det återstår 
% därför att bestämma vad vi menar med $e^{iy}$. Vi kan inte utgå från att $e^{iy}$ 
% är ett reellt tal, utan vi måste anta att den är av formen $e^{iy} = f(y)+ig(y)$ 
% där $f$ och $g$ är reellvärda funktioner. Vill vi då att punkt 3 skall vara 
% uppfylld må ste vi alltså ha 
% \begin{equation*}
% D(e^{iy}) = f'(y) + ig'(y) = ie^{iy} \Leftrightarrow f(y) + ig(y) = g'(y) - if'(y)
% \end{equation*}
% dvs 
% \begin{equation*}
% f(y) = g'(y) \ \text{och} \ g(y) = -f'(y).
% \end{equation*}
% Deriverar vi den högra likheten en gång till får vi $f(y) = -f''(y)$. 
% Eftersom vi vill att även punkt 2 skall vara uppfylld måste vi ha $f(0) = 1$ 
% och $g(0) = f'(0) = 0$. Två
% funktioner som uppfyller dessa villkor är $f(y) = \cos y$ och $g(y) = \sin y$
% och vi sätter därför:

\begin{definition}
\label{def_e}
Om $z=a+ib$, där $a,b\in\R$, så sätter vi 
\begin{equation*}
e^z = e^{a + ib} = e^ae^{ib} = e^a(\cos b + i\sin b).
\end{equation*}
\end{definition}
\onslide<2->
\begin{itemize}
\item $e^z$ överensstämmer med den reella exponentialfunktionen om $z\in \R$ 
\item Ett komplext tal på polär form kan nu skrivas som
% \begin{equation*}
%     z = r(\cos \theta + i\sin \theta) = re^{i\theta }
% \end{equation*}
\end{itemize}
\begin{center}
\begin{minipage}{0.35\linewidth}
\begin{block}{}
$z = r(\cos \theta + i\sin \theta) = re^{i\theta }$
\end{block}  
\end{minipage}
\end{center}
\onslide<5->


Definition \ref{def_e} ger: 
\begin{center}
\begin{minipage}{0.75\textwidth}
\begin{varblock}[0.75\textwidth]{Eulers formler}
$\dst\cos \theta = \frac{e^{i\theta } + e^{-i\theta }}{2} \qquad
\sin \theta = \frac{e^{i\theta } - e^{-i\theta }}{2i}$
\end{varblock}
\end{minipage}
\end{center}

% \begin{exempel}
% Visa Eulers formler.
% \end{exempel} 
\end{frame}

%-------------------------------------------------------------------------------
\begin{frame}
\frametitle{Den komplexa exponentialfunktionen}

\begin{sats}[Potenslagar]
För två godtyckliga komplexa tal $z$, $z_1$ och $z_2$ gäller

\begin{enumerate}
\item  $e^{z_1}e^{z_2} = e^{z_1 + z_2}$.
\item  $\displaystyle e^{-z} = \frac{1}{e^z}$.
\item  $(e^z)^n = e^{nz}$, där $n$ är ett heltal 
(\emph{de Moivres formel}).
\end{enumerate}
\end{sats}

% %\begin{Bevis}
% \begin{enumerate}
% \item  För $z_1 = a + ib$ och $z_2 = c + id$ har vi 
% \begin{align*}
% e^{z_1}e^{z_2} &= e^a(\cos b + i\sin b)e^c(\cos d + i\sin d) \\
% & = e^{a+c}[(\cos b\cos d - \sin b\sin d) + i(\sin b\cos d+\cos b\sin d)] \\
% & = e^{a+c}(\cos (b+d) + i\sin (b+d)) = e^{(a+c)+i(b+d)} = e^{a+ib+c+id} \\
% & = e^{z_1 + z_2}.
% \end{align*}

% \item  Använder vi potenslag 1 så får vi 
% \begin{equation*}
% e^ze^{-z} = e^{z+(-z)} = e^{0} = 1 \Rightarrow e^{-z} =\frac{1}{e^z}.
% \end{equation*}

% \item  Om $n=0$ är likheten definitionsmässigt uppfylld. Om $n$ är ett 
% positivt heltal så gäller: 
% \begin{equation*}
% (e^z)^n = e^z\cdot e^z \cdots e^z = e^{z + z + ... + z} = e^{nz}.
% \end{equation*}
% Om $n$ är ett negativt heltal så sätter vi $m=-n$. Eftersom $m>0$ så har vi 
% enligt potenslag 2: 
% \begin{equation*}
% (e^z)^n = (e^z)^{-m} = \frac{1}{(e^z)^m} = \frac{1}{e^{mz}} = e^{-mz} = e^{nz}.
% \end{equation*}
% \end{enumerate}
% %\end{Bevis}
\onslide<+->
Om $z_1 = r_1e^{i\theta_1}$ och $z_2 = r_2e^{i\theta_2}$ får vi 
\onslide<+->
\begin{eqnarray*}
z_1\cdot z_2& =& r_1e^{i\theta_1}r_2e^{i\theta_2} = \onslide<+-> r_1r_2e^{i(\theta_{1}+\theta_{2})} \\
\onslide<+-> \frac{z_{1}}{z_{2}}& =&\frac{r_{1}e^{i\theta _{1}}}{r_{2}e^{i\theta _{2}}}=
\onslide<+->
\frac{r_{1}}{r_{2}}e^{i\theta _{1}}e^{-i\theta _{2}}
\onslide<+-> =\frac{r_{1}}{r_{2}}e^{i(\theta _{1}-\theta_{2})}
\end{eqnarray*}
\onslide<+->
Vid {\color{hhgreen}multiplikation}/{\color{hhblue1}division} av två komplexa tal i
polär form:
\begin{itemize}
\item {\color{hhgreen}multipliceras}/{\color{hhblue1}divideras}
 absolutbeloppen
\item {\color{hhgreen}adderas}/{\color{hhblue1}subtraheras} argumenten
\end{itemize}
\end{frame}

%-------------------------------------------------------------------------------
\begin{frame}
\frametitle{Den komplexa exponentialfunktionen}
% Om nu $z_1 = r_1e^{i\theta_1}$ och $z_2 = r_2e^{i\theta_2}$ så får vi 
% \begin{align*}
% z_1\cdot z_2& = r_1e^{i\theta_1}r_2e^{i\theta_2} = r_{1}r_{2}e^{i(\theta _{1}+\theta _{2})} \\
% \frac{z_{1}}{z_{2}}& =\frac{r_{1}e^{i\theta _{1}}}{r_{2}e^{i\theta _{2}}}=%
% \frac{r_{1}}{r_{2}}e^{i\theta _{1}}e^{-i\theta _{2}}=\frac{r_{1}}{r_{2}}%
% e^{i\theta _{1}}e^{-i\theta _{2}}=\frac{r_{1}}{r_{2}}e^{i(\theta _{1}-\theta
% _{2})}.
% \end{align*}
% Vi ser alltså att vid \emph{multiplikation} av två komplexa tal i
% polär form så \emph{multipliceras absolutbeloppen} och \emph{adderas
% argumenten}. På motsvarande sätt få r vi att vid \emph{division} av
% två komplexa tal i polär form så \emph{divideras absolutbeloppen}
% och \emph{subtraheras argumenten}.

\begin{exempel}
Om $\displaystyle z_{1}=2e^{i\frac{\pi }{3}}$ och $\displaystyle z_{2}=3e^{i%
\frac{\pi }{4}}$ vad blir $\displaystyle z_{1}\cdot z_{2}$ och $%
\displaystyle 
\frac{z_{1}}{z_{2}}$ ?
\onslide<+->
\begin{losning}
\begin{eqnarray*}
z_{1}\cdot z_{2}& =&\onslide<+-> 2\cdot 3e^{i(\frac{\pi }{3}+\frac{\pi }{4})}= \onslide<+->
6e^{i\frac{7\pi }{12}} \\
\onslide<+->
\frac{z_{1}}{z_{2}}& =&\onslide<+-> \frac{2}{3}e^{i(\frac{\pi }{3}-\frac{\pi }{4})}=
\onslide<+->
\frac{2}{3}e^{i\frac{\pi }{12}}
\end{eqnarray*}
\end{losning}
\end{exempel}
\begin{exempel}
Skriv talet $\left( 1+i\right)^{24}$ på rektangulär form.
\onslide<+->
\begin{losning}
$1+i=\onslide<+-> \sqrt{2}e^{i\frac{\pi }{4}} \Rightarrow  
\onslide<+->
\left(1+i\right)^{24}=\onslide<+->\left( \sqrt{2}e^{i\frac{\pi }{4}}\right) ^{24}=
\onslide<+->\sqrt{2}^{24}e^{i\frac{24\pi }{4}}=\onslide<+->2^{12}e^{i6\pi }=\onslide<+->2^{12} = 4096$
\end{losning}
\end{exempel}
\end{frame}

% %-------------------------------------------------------------------------------
% \begin{frame}
% \frametitle{Den komplexa exponentialfunktionen}
% \begin{exempel}
% Skriv talet $\left( 1+i\right)^{24}$ på rektangulär form.
% \onslide<2->
% \begin{losning}
% $1+i=\sqrt{2}e^{i\frac{\pi }{4}} \Rightarrow  
% \onslide<3->
% \left(1+i\right)
% ^{24}=\left( \sqrt{2}e^{i\frac{\pi }{4}}\right) ^{24}=\sqrt{2}^{24}e^{i\frac{%
% 24\pi }{4}}=2^{12}e^{i6\pi }=2^{12}$
% \end{losning}
% \end{exempel}
% \onslide<4->
% \end{frame}

%-------------------------------------------------------------------------------
\begin{frame}
\frametitle{Den komplexa exponentialfunktionen}
\begin{exempel}
Förenkla $z=\dst \frac{i(\sqrt{3}-i)^3}{(-1+i)^2}$. Ange svaret på
rektangulär och polär form.
\onslide<+->
\begin{losning}
\begin{eqnarray*}
|z| &=& \frac{|i|\cdot|\sqrt{3} -i|^3}{|-1+i|^2} = \onslide<+->
\frac{1\cdot 2^3}{\sqrt{2}^2} = \onslide<+->
4\\
\onslide<+->
\arg z &=& \onslide<+-> \arg(i) + 3\arg (\sqrt{3}-i) -2\arg{(-1+i}) = 
\onslide<+->\frac{\pi}{2}
\onslide<+->+3(-\frac{\pi}{6})\onslide<+->-2\frac{3\pi}{4} = \onslide<+->-\frac{3\pi}{2}\\
\onslide<+->
\Rightarrow z &=&\onslide<+-> 4e^{i\frac{\pi}{2}} = \onslide<+-> =
4(\cos \frac{\pi}{2} + i\sin \frac{\pi}{2}) = \onslide<+-> 4i 
\end{eqnarray*}
\end{losning}
\end{exempel}
\end{frame}

%-------------------------------------------------------------------------------
\begin{frame}
\frametitle{Den komplexa exponentialfunktionen}
\begin{exempel}
Vad innebär multiplikation och division med talet $i$ för den
grafiska tolkningen av komplexa tal?
\onslide<+->
\begin{losning}
\vspace{-0.8cm}
\begin{columns}[c]
\column{6cm}    
$z=re^{i\theta }$ och $i=e^{i\frac{\pi }{2}} \Rightarrow$  
\begin{eqnarray*}
\onslide<+->
iz& =&re^{i\theta }e^{i\frac{\pi }{2}}\onslide<+-> =re^{i(\theta +\frac{\pi }{2})} \\
\onslide<+->
\frac{z}{i}& =&\frac{re^{i\theta }}{e^{i\frac{\pi }{2}}}=\onslide<+->
re^{i\theta }e^{-i\frac{\pi }{2}}=\onslide<+->
re^{i(\theta -\frac{\pi }{2})}
\end{eqnarray*}
\onslide<+->
\column{3.7cm}
\hspace{-1cm}
\includegraphics[width=4cm]{bilder/komplexmulti1.pdf}
\end{columns}
\onslide<+->
% Multiplikation med $i$ motsvaras av en vridning moturs av
% vektorn $z$ vinkeln $\frac{\pi }{2}$ och division med $i$ av en motsvarande
% vridning medurs.
\vspace{0.3cm}
{\color{hhgreen}Multiplikation}/{\color{hhred}division} med $i$ motsvaras av en vridning {\color{hhgreen} moturs}/{\color{hhred}medurs} av
vektorn $z$ vinkeln $\frac{\pi}{2}$.

\end{losning}
\end{exempel}

% Allmänt gäller att om man multiplicerar ett komplext tal $z_{1}$ med
% ett annat komplext tal $z_{2}$ så motsvaras detta av att vektorn $z_{1}$
% vrids en vinkel lika med argumentet för $z_{2}$ och förändras
% med en faktor lika med beloppet av $z_{2}$ (Fig. 7).
\end{frame}

% \section{Något om kvadratrötter}

% Kvadratroten ur ett tal $a$ är ett tal $x$ så dant att 
% \begin{equation*}
% x^{2}=a.
% \end{equation*}
% Observera att denna ekvation har \emph{två } lösningar. Som ett
% exempel har ekvationen $x^{2}=25$ lösningarna $x=± 5$. Normalt (bl a i
% skolan) följer man konventionen att om $a$ är ett \emph{positivt
% reellt tal} så lå ter man $\sqrt{a}$ beteckna den \emph{positiva}
% lösningen. T ex $\sqrt{25}=5$ dvs det \emph{positiva} reella tal som i
% kvadrat blir $25$. %Ekvationen $x^2=25$  har alltså rötterna 
% %$x=± \sqrt{25}=± 5$.

% Om talet $a$ i ekvation \eqref{rot} är \emph{negativt} eller t om \emph{%
% komplext} (med imaginärdel skild frå n noll) blir kvadratroten inget
% reellt tal och man kan därför inte tala om den positiva
% lösningen. I dessa fall lå ter man $\sqrt{a}$ beteckna \emph{bå da}
% lösningarna. T ex $\sqrt{-1}$ betyder ett tal $x$ så dant att $x^2 =
% -1 $. Denna ekvation har lösningen $x = ± i$ och vi har alltså att $%
% \sqrt{-1} = ± i$. Man bör därför undvika att som i vissa
% läroböcker använda beteckningen $i = \sqrt{-1}$ för den
% imaginära enheten.

% Det vanligt förekommande ``beviset'' för att $1=-1$ dvs 
% \begin{equation*}
% 1=\sqrt{1}=\sqrt{(-1)\cdot (-1)}=\sqrt{-1}\sqrt{-1}=i\cdot i=i^{2}=-1
% \end{equation*}
% är alltså felaktigt därför att $\sqrt{-1}\sqrt{-1}=(±
% i)\cdot (± i)=± 1$ inte är entydigt. I ``beviset'' ovan må ste
% alltså $\sqrt{-1}\sqrt{-1}$ vara lika med $+1$ medan det i nå got annat
% fall lika gärna kan vara lika med $-1$. Lagen 
% \begin{equation*}
% \sqrt{ab}=\sqrt{a}\sqrt{b}
% \end{equation*}
% gäller alltså endast för positiva reella tal $a$ och $b$.

% Sensmoralen i detta resonemang är att man bör vara \emph{mycket}
% försiktig om man under rottecknet har nå gonting annat än ett
% positivt reellt tal.

% \begin{Ovning}
% Ange på formen $a+ib$

% \begin{enumerate}
% \item[(a)]  {$\sqrt{-4}$}

% \item[(b)]  {$\sqrt{i}$}

% \item[(c)]  {$\sqrt{\frac{1-i\sqrt{3}}{2}}$}
% \end{enumerate}
% \end{Ovning}

% \noindent Ledning: Ett komplext tal $z$ kan skrivas på polär form: $%
% z=re^{i(\theta +n\cdot 2\pi )}$ där $n$ är ett godtyckligt heltal
% och $\sqrt{z}=z^{\frac{1}{2}}$.

%\subsection{Binomiska ekvationer}
%-------------------------------------------------------------------------------
\begin{frame}
\frametitle{Binomiska ekvationer}


\begin{exempel}
Lös ekvationen $z^{2}=2i$.
\onslide<+->
\begin{losning}
\begin{eqnarray*}
z&=&a+ib \onslide<+-> \Rightarrow z^2=(a+ib)^{2}=\onslide<+->
 a^{2}-b^{2}+2abi=\onslide<+-> 2i\\
\onslide<+->
&\Leftrightarrow& \left\{ 
\begin{array}{llll}
a^{2}-b^{2} & =\onslide<+->
 & 0 & (1) \\
\onslide<+->
 2ab & = \onslide<+->
& 2 & (2)
\end{array}
\right.
\end{eqnarray*}
\newline
\onslide<+->
(2) $\Rightarrow a=1/b$. \onslide<+-> Insättning i (1) $\Rightarrow
b^{4}=1\Leftrightarrow \onslide<+->
b=\pm 1\Rightarrow \onslide<+->
a=\pm 1$. \\
\onslide<+->
Enligt (2) har $a$ och $b$ har samma tecken och vi får lösningarna $z=\onslide<+->
\pm (1+i)$.
\end{losning}
\end{exempel}
\begin{itemize}
\item Ekvationen $z^{n}=w$, där $w \in \mathbb{C}$ och $n\in \mathbb{Z}$, kallas en
\alert{binomisk ekvation}.
\item För högre $n>2$ blir det jobbigt att lösa binomiska
 ekvationer med metoden ovan. Det är bättre att gå över till polär form.
\end{itemize}

\end{frame}

%-------------------------------------------------------------------------------
\begin{frame}
\frametitle{Binomiska ekvationer}
\begin{exempel}
Lös ekvationen $z^{3}=8i$.
\onslide<+->
\begin{losning}
\vspace{-0.6cm}
\begin{columns}[c]
\column{0.6\textwidth}
\begin{eqnarray*}
z&=&re^{i\theta } \text{ och } 8i=\onslide<+->8e^{i\frac{\pi }{2}} \Rightarrow  
\onslide<+->
z^{3}=r^{3}e^{i3\theta }=\onslide<+->8e^{i\frac{\pi }{2}}\\
\onslide<+->
&\Leftrightarrow&
\begin{cases}
r=8^{1/3}=\onslide<+->2\\
\onslide<+->
3\theta = \onslide<+->\frac{\pi }{2}\onslide<+->+k\cdot 2\pi \onslide<+->\Rightarrow \theta =\frac{\pi }{6} +k\cdot \frac{2\pi }{3}
\end{cases}
\end{eqnarray*}
där $k$ är ett godtyckligt heltal. 
\column{0.3\textwidth}
\only<34->{\includegraphics[width=3cm]{bilder/komplexbinom1.pdf}}
\end{columns}
\begin{eqnarray*}
\onslide<+->
k &=&0:\onslide<+->
\quad z_{0}=2e^{i\frac{\pi }{6}}=\onslide<+->
2(\cos \frac{\pi }{6}+i\sin\frac{\pi }{6})=
\onslide<+->
2(\frac{\sqrt{3}}{2}+\onslide<+->i\frac{1}{2})=\onslide<+->\sqrt{3}+i   \\
\onslide<+->
k &=&1:\onslide<+->
\quad z_{1}=2e^{i\frac{5\pi }{6}}=\onslide<+->
2(\cos \frac{5\pi }{6}+i\sin\frac{5\pi }{6})=
\onslide<+->
2(-\frac{\sqrt{3}}{2}+\onslide<+->i\frac{1}{2})=\onslide<+->-\sqrt{3}+i   \\
\onslide<+->
k &=&2:\onslide<+->
\quad z_{2}=2e^{i\frac{3\pi }{2}}=\onslide<+->
2(\cos \frac{3\pi }{2}+i\sin\frac{3\pi }{2})=
\onslide<+->2(0-i)=-2i  \\
\onslide<+->
k &=&3:\onslide<+->
\quad z_{3}=2e^{i\frac{\pi }{6}}=\onslide<+->z_{0}  
\end{eqnarray*}
%dvs tre \textit{olika} rötter.
\end{losning}
\end{exempel}
\end{frame}

% %-------------------------------------------------------------------------------
% \begin{frame}
% \frametitle{Binomiska ekvationer}
% Allmänna fallet $z^n=w$: $z=re^{i\theta }$ och $w=\rho
% e^{i\varphi } \Rightarrow$ 
% \onslide<+->
% \begin{eqnarray*}
% \onslide<+->
% z^{n}&=&(re^{i\theta })^{n}=r^{n}e^{in\theta }=w=\rho e^{i\varphi}
% \onslide<+->
% \Leftrightarrow \left\{ 
% \begin{array}{ll}
% r^{n}=\onslide<+->\rho &  \\ 
% n\theta =\onslide<+-> \varphi +k\cdot 2\pi , & k\in \mathbb{Z}
% \end{array}
% \right.\\
% \Leftrightarrow 
% \onslide<+->
% z_{k}&=&\rho ^{1/n}e^{i(\frac{\varphi }{n}+k\cdot \frac{2\pi }{n})},\, k\in \mathbb{Z}.
% \end{eqnarray*}
% \onslide<+->
% \alert{Anm:} 
% \begin{equation*}
% z_{k=n}=\rho ^{1/n}e^{i(\frac{\varphi }{n}+n\cdot \frac{2\pi }{n})}=\onslide<+->
% \rho^{1/n}e^{i(\frac{\varphi }{n}+2\pi )}=\onslide<+->
% \rho ^{1/n}e^{i\frac{\varphi }{n}}=\onslide<+->z_{0}
% \end{equation*}
% Ekvationen $z^n=w$ har alltså precis $n$ st olika lösningar.
% \begin{sats}[Binomiska ekvationer]
% Den binomiska ekvationen 
% %\begin{equation*}
% $z^{n}=w=\rho e^{i\theta }$
% %\end{equation*}
% har rötterna 
% \begin{equation*}
% z_{k}=\rho ^{1/n}e^{i(\frac{\theta }{n}+k\cdot \frac{2\pi }{n}%
% )},\,k=0,1,...,n-1
% \end{equation*}
% \end{sats}
% \begin{anm}
% $|z_{k}|=\rho ^{1/n}$ och vinkeln mellan
% två närliggande rötter är $\frac{2\pi}{n}$ \\$\Rightarrow$
% rötterna bildar hörn i 
% en regelbunden $n$-hörning inskriven i en cirkel med medelpunkt i origo och 
% radie $\rho^{1/n}$.  
% \end{anm}
% \end{frame}

% %-------------------------------------------------------------------------------
% \begin{frame}
% \frametitle{Binomiska ekvationer}
% \begin{exempel}
% Lös ekvationen $z^{8}=1-\sqrt{3}i$ och rita in rötterna i det komplexa talplanet.
% \onslide<+->
% \begin{losning}
% \vspace{-0.8cm} 
% \begin{columns}[c]
% \column{0.7\textwidth}    
% \begin{eqnarray*}
% 1-\sqrt{3}i&=\onslide<+->&2e^{i\frac{5\pi }{3}} \\
% \onslide<+->
% \Rightarrow
% z_{k}&=&2^{1/8}e^{i(\frac{5\pi }{24}+k\cdot \frac{\pi }{4})},\, k=0,1,2,...,7.
% \end{eqnarray*}
% \column{0.3\textwidth}
% \only<4->{\includegraphics[width=3cm]{bilder/komplexbinom2.pdf}}
% \end{columns}
% \end{losning}
% \end{exempel}

% \vspace{5cm}
% \end{frame}

% %-------------------------------------------------------------------------------
% \begin{frame}
% \frametitle{Något mer om rötter}
% \begin{itemize}
% \item $x^2 = 4 \Leftrightarrow x = \pm 2$. \onslide<+->
% Den \alert{positiva roten} 2 betecknas $\sqrt{4}$
% % \[
% % \sqrt{4}=\{\text{ Det positiva reella tal som i kvadrat blir 4}\} = 2.
% % \]
% \item Kan vi definiera $\sqrt{z}$ entydigt på motsvarande sätt? 
% \onslide<+->
% %Den binomiska ekvationen $x^n = z=r e^{i\theta }$ har rötterna
% \[
% x^n = z=r e^{i\theta } \text{ har rötterna } x_k=r^{1/n}e^{i(\frac{\theta}{n}+k\cdot
%     \frac{2\pi}{n})},\,k=0,1,...,n-1
% \]
% \onslide<+->
% Om $-\pi < \theta \leq \pi$ (principalargumentet till $z$) kan
%   man definiera \alert{principalroten} som $x_0$:
% \vspace{-0.3cm}
% \begin{center}
% \begin{minipage}{0.3\linewidth}
% \begin{block}{}
% $\sqrt[n]{z}=z^{1/n}=r^{1/n}e^{i\frac{\theta}{n}}$
% \end{block}  
% \end{minipage}
% \end{center}
% \item Ex: Kvadratroten av $-1$ (principalroten):
% \onslide<+->
% \[
% -1=\onslide<+->
% e^{i\pi} \onslide<+->
% \Rightarrow \sqrt{-1} = \onslide<+->
% e^{i\frac{\pi}{2}} = \onslide<+->i
% \]
% \onslide<+->
% Ekvationen $z^2+1=0$ har alltså rötterna $z=\pm\sqrt{-1} = \pm i$
% \end{itemize}
% \onslide<+->
% \begin{alertblock}{Varning!}
% \[
% \onslide<+->1=\sqrt{1}=
% \onslide<+->\sqrt{(-1)(-1)} = 
% \onslide<+->\sqrt{-1}\sqrt{-1} = 
% \onslide<+->i\cdot i = 
% \onslide<+->i^2 =
% \onslide<+->  -1 \quad?!?
%   \]
% \onslide<+->
%   $\sqrt{ab}=\sqrt{a}\sqrt{b}$ gäller generellt endast om
%   $a$ och $b$ är icke-negativa realla tal!
% \end{alertblock}
% \end{frame}


% %\subsection{Andragradsekvationer med komplexa koefficienter}
% %-------------------------------------------------------------------------------
% \begin{frame}
% \frametitle{Andragradsekvationer med komplexa koefficienter}
% % För andragradsekvationer med reella koefficienter finns en enkel formel
% % för rötterna. Denna formel kan inte användas för
% % andragradsekvationer med komplexa koefficienter eftersom vi inte definierat
% % kvadratroten ur ett komplext tal.
% \onslide<+->
% \begin{exempel}
% Lös ekvationen $iz^{2}+(2-3i)z-1+5i=0$.
% \onslide<+->
% \begin{losning}
% \begin{eqnarray*}
% z^{2}+\frac{2-3i}{i}z-\frac{1-5i}{i} &=\onslide<+->
% &z^{2}-(2i+3)z+i+5  \notag \\
% &=&\onslide<+->
% \left( z-\frac{3+2i}{2}\right) ^{2}-\left( \frac{3+2i}{2}\right)
% ^{2}+5+i=0  \notag \\
% \onslide<+->
% \Leftrightarrow \left( z-\frac{3+2i}{2}\right)^{2} &=&\onslide<+->
% -(5+i)+\left( \frac{3+2i}{2}\right) ^{2}=\onslide<+->
% -\frac{15}{4}+2i 
% \end{eqnarray*}
% \onslide<+->
% Sätt $\displaystyle z-\frac{3+2i}{2}=x+iy,\, x,y\in\R\, \onslide<+->
% \Rightarrow (x+iy)^{2}=\onslide<+->
% x^{2}-y^{2}+2xyi=\onslide<+->
% -\frac{15}{4}+2i$
% \onslide<+->
% \begin{equation*}
% \Leftrightarrow \left\{ 
% \begin{array}{ll}
% x^{2}-y^{2}=\onslide<+->
% \displaystyle-\frac{15}{4} & (1) \\ 
% \onslide<+->
% xy=\onslide<+->
% 1 & (2)
% \end{array}
% \right.
% \end{equation*}
% \end{losning} 
% \end{exempel}
% \end{frame}

% %-------------------------------------------------------------------------------
% \begin{frame}
% \frametitle{Andragradsekvationer med komplexa koefficienter}
% \addtocounter{exempel}{-1}
% \begin{exempel}[forts]
% \onslide<+->
% (2) $\Rightarrow \displaystyle x=\frac{1}{y} \,$ 
% \onslide<+->
% Insättning i (1): 
% \begin{eqnarray*}
% \frac{1}{y^{2}}-y^{2}=\onslide<+->
% -\frac{15}{4} 
% \onslide<+->
% &\Leftrightarrow &y^{4}-\frac{15}{4}%
% y^{2}-1=0  \notag \\
% \onslide<+->
% &\Leftrightarrow &y^{2}=\onslide<+->
% \frac{15}{8}\pm \sqrt{\left( \frac{15}{8}\right)^{2}+1}\onslide<+->\underset{y^2\geq 0}{=}
% \frac{15}{8} + \frac{17}{8} = 4 \notag
% \onslide<+->
% \end{eqnarray*}
% $x$ och $y$ är reella och har samma tecken enligt (2):
% \onslide<+->
% \begin{equation*}
% \Rightarrow y=\pm 2,\, x=\pm \frac{1}{2}
% \end{equation*}
% \begin{equation*}
% \onslide<+->
% \Rightarrow z-\frac{3+2i}{2}=\onslide<+->
% \pm \left( \frac{1}{2}+2i\right) \onslide<+->
% \Leftrightarrow z=\frac{3+2i}{2}\pm \left( \frac{1}{2}+2i\right) 
% \onslide<+->
% \end{equation*}
% $\therefore z_{1}=2+3i$ och $z_{2}=1-i$
% \end{exempel}
% \end{frame}

% \section{Algebraiska ekvationer}
% \subsection{Polynom och algebraiska ekvationer}
% %===============================================================================
% %-------------------------------------------------------------------------------
% \begin{frame}
% \frametitle{Polynom och algebraiska ekvationer}
% % \chapter{Polynom och algebraiska ekvationer}
% \begin{itemize}
% \item Vi skall nu studera allmänna \alert{algebraiska ekvationer}:
% \begin{center}
% \begin{minipage}[c]{0.5\linewidth}
% \begin{block}{}
%        $a_{n}z^{n}+a_{n-1}z^{n-1}+...+a_{1}z+a_{0}=0$
% \end{block}
% \end{minipage}
% \end{center}
% \onslide<+->
% där $a_{n},a_{n-1},...,a_{1},a_{0}\in \mathbb{C}$.
% %\item Ex: 2:a-gradsekvationer och binomiska ekvationer. 
% \item För ekvationer av grad $\leq 4$ finns formler för rötterna. 
% \item För $n\geq 5$ går det inte att bestämma rötterna med en formel.%algebraiska operationer.
% %kan det inte finnas en sluten formel för rötterna. 
% % Abel showed that there is no general algebraic solution for the roots
% % of a quintic equation, or any general polynomial equation of degree
% % greater than four, in terms of explicit algebraic operations.
% \end{itemize}
% \onslide<+->
% \begin{anm}
% \begin{columns}[c]
% \column{0.78\textwidth}
% %Lösningsformler för allmänna algebraiska ekvationer av grad $n\geq 5$ hade matematiker arbetat %med i 250 år
% %när 
% Niels Henrik Abel (1802-1829) bevisade att det
% inte går att bestämma rötterna till allmänna algebraiska ekvationer 
% av
% grad $\geq 5$ med algebraiska operationer, dvs med en formel som utgår
% från ekvationens koefficienter (som $pq$-formeln för $n=2$). %med algebraiska operationer dvs 
% \column{0.15\textwidth}
% %\includegraphics[width=1.7cm]{bilder/Niels_Henrik_Abel.jpg}
% \shadowimage[width=1.7cm]{bilder/Niels_Henrik_Abel.jpg}
% \end{columns}
% \end{anm}
% \end{frame} 

% %===============================================================================
% %-------------------------------------------------------------------------------
% \begin{frame}
% \frametitle{Polynom och algebraiska ekvationer}
% % \section{Grundläggande definitioner}

% \begin{definition}[Polynom]
% En funktion 
% \begin{equation*}
% p(z)=a_{n}z^{n}+a_{n-1}z^{n-1}+...+a_{1}z+a_{0}
% \end{equation*}
% där $a_{n},a_{n-1},...,a_{1},a_{0}\in \mathbb{C}$ och $n\in \mathbb{N}$
% kallas en \textit{polynomfunktion} eller ett \textit{polynom}. Om $a_{n}\neq
% 0$ har polynomet grad $n$.
% \end{definition}

% \begin{exempel}
% $p(z) = 5z^4 - 3iz^2 + (2 - 4i)z - 2 + i$ är ett polynom av grad 4.
% \end{exempel}

% %Delbarhetsegenskaperna hos polynom ges av

% \begin{sats}[Divisionsalgoritmen]
% Om $p$ och $f$ är två polynom och $f\neq 0$ så finns det polynom $q$ och $r$ sådana att 
% \begin{equation*}
% p(z)=f(z)\underset{\text{kvot}}{q(z)}+\underset{\text{rest}}{r(z)}
% \text{ \quad med grad } r < \text{ grad } f.% eller $r=0$.
% \end{equation*}
% %där grad $r<$ grad $f$.% eller $r=0$.
% \label{divalg}
% \end{sats}

% % \begin{anm}
% % $q$ = kvot, $r$ = rest.
% % \end{anm}
% \end{frame}

% %-------------------------------------------------------------------------------
% \begin{frame}
% \frametitle{Polynom och algebraiska ekvationer}
% % Vi gör inget allmänt bevis för divisionsalgoritmen utan
% % illustrerar satsen med ett exempel.

% \begin{exempel}
% Ange kvoten och resten då $p(z) = z^3 - 2z^2 + z + 1$ divideras
% med $f(z) = z^2 + z$.
% \onslide<+->
% \begin{losning}
% Polynomdivision:
% \onslide<+->
% \[
% {\small \polylongdiv[vars=z,style=A]{z^3-2z^2+z+1}{z^2+z} }
% \]
% \onslide<+->
% \[
% \Rightarrow p(z) = z^3-2z^2+z+1 = (z^2+z)(z-3) + 4z+1
% \]
% \onslide<+->
% $\therefore$ Kvoten är $q(z)=z-3$ och resten $r(z) =4z+1$ \\
% \end{losning}

% % \begin{equation*}
% % \begin{tabular}{ccccccccccc}
% % &  &  &  & $z$ & $-$ & 3 &  &  &  & $z^2$ \\ 
% % &  &  &  &  &  &  &  &  &  &  \\ 
% % $z^2$ & $+$ & $z$ &  & $z^3$ & $-$ & $2z^2$ & $+$ & $z$ & $+$ & 1 \\ 
% % &  &  &  &  &  &  &  &  &  &  \\ 
% % &  &  & $-$ & $(z^3$ & $+$ & $z^2)$ &  &  &  &  \\ 
% % &  &  &  &  &  &  &  &  &  &  \\ 
% % &  &  &  &  &  & $-3z^2$ & $+$ & $z$ & $+$ & 1 \\ 
% % &  &  &  &  &  &  &  &  &  &  \\ 
% % &  &  &  &  & $-$ & $(-3z^2$ & $-$ & $3z)$ &  &  \\ 
% % &  &  &  &  &  &  &  &  &  &  \\ 
% % &  &  &  &  &  &  &  & $4z$ & $+$ & 1 \\ 
% % &  &  &  &  &  &  &  &  &  & 
% % \end{tabular}
% % \end{equation*}
% % Vi får kvoten $q(z) = z - 3$ och resten $r(z) = 4z + 1$, dvs $p(z) = (z^2
% % + z)(z-3) + 4z + 1$ och vi ser att grad $r = 1 < $ grad $f$ = 2.
% \end{exempel}
% \begin{anm}
% grad $r$ = 1 < grad $f$ = 2.  
% \end{anm}
% \end{frame}

% %\subsection{Faktorsatsen}
% %-------------------------------------------------------------------------------
% \begin{frame}
% \frametitle{Faktorsatsen}
% % \section{Algebraiska ekvationer, rötter och nollställen}

% \begin{definition}
% En \textit{algebraisk ekvation} är en ekvation av typen 
% \begin{equation*}
% p(z)=a_{n}z^{n}+a_{n-1}z^{n-1}+...+a_{1}z+a_{0}=0.
% \end{equation*}
% Ett tal $\alpha \in \mathbb{C}$ kallas ett \textit{nollställe} till
% polynomet $p(z)$ eller en \textit{rot} till ekvationen $p(z)=0$ om $p(\alpha
% )=0$. 
% \end{definition}
% \onslide<2->
% Sambandet mellan nollställen till och faktorisering av polynom
% ges av:
% \begin{sats}[Faktorsatsen]
% Om $p(z)$ är ett polynom så gäller 
% \begin{equation*}
% p(\alpha )=0\Leftrightarrow p(z)=q(z)(z-\alpha )
% \end{equation*}
% där $q(z)$ är ett polynom med grad $q$ = grad $p-1$.
% \label{faktsats}
% \end{sats}
 
% % \begin{Bevis}

% % \begin{enumerate}
% % \item  ($\Rightarrow $): Dividerar vi $p(z)$ med $z-\alpha $ så får vi 
% % \begin{equation*}
% % p(z)=q(z)(z-\alpha )+r(z)\text{ }(\ast )
% % \end{equation*}
% % där $r$ antingen är noll eller ett polynom vars grad är mindre
% % än grad($z-\alpha $) = 1 enligt divisionsalgoritmen, dvs en konstant $%
% % C\neq 0$. Sätt nu $z=\alpha $ i $(\ast )$: $0=p(\alpha )=q(\alpha
% % )(\alpha -\alpha )+C=C\Rightarrow p(z)=q(z)(z-\alpha ).$

% % \item  ($\Leftarrow $): 
% % \begin{equation*}
% % p(z)=q(z)(z-\alpha )\Rightarrow p(\alpha )=q(\alpha )(\alpha -\alpha )=0.
% % \end{equation*}
% % $\alpha $ är alltså ett nollställe till $p(z)$.$%
% % \hfill $
% % \end{enumerate}
% % \end{Bevis}
% \end{frame}

% %-------------------------------------------------------------------------------
% \begin{frame}
% \frametitle{Faktorsatsen}
% \begin{exempel}
% Ekvationen $p(z)=z^{3}-(3+i)z^2+(2+3i)z-2i=0$ har en rot $z=i$. Lös
% ekvationen.
% \onslide<+->
% \begin{losning}
% Enligt faktorsatsen är $z-i$ en faktor i $p(z)$. 
% \onslide<+-> 
% Polynomdivision ger:
% \onslide<+->
% %{\small \polylongdiv[style=A]{x^{3}-(3+i)x+(2+3i)x-2i}{x-i} }
% % \begin{equation*}
% % \begin{tabular}{ccccccccccc}
% % &  &  &  & $z^{2}$ & $-$ & $3z$ & $+$ & $2$ & $z$ & $-$ \\ 
% % &  &  &  &  &  &  &  &  &  &  \\ 
% % $z$ & $-$ & $i$ &  & $z^{3}$ & $-$ & $(3+i)z^{2}$ & $+$ & $(2+3i)z$ & $-$ & $%
% % 2i$ \\ 
% % &  &  &  &  &  &  &  &  &  &  \\ 
% % &  &  & $-$ & $(z^{3}$ & $-$ & $iz^{2})$ &  &  &  &  \\ 
% % &  &  &  &  &  &  &  &  &  &  \\ 
% % &  &  &  &  &  & $-3z^{2}$ & $+$ & $(2+3i)z$ & $-$ & $2i$ \\ 
% % &  &  &  &  &  &  &  &  &  &  \\ 
% % &  &  &  &  & $-$ & $(-3z^{2}$ & $+$ & $3iz)$ &  &  \\ 
% % &  &  &  &  &  &  &  &  &  &  \\ 
% % &  &  &  &  &  &  &  & $2z$ & $-$ & $2i$ \\ 
% % &  &  &  &  &  &  &  &  &  &  \\ 
% % &  &  &  &  &  &  & $-$ & $(2z$ & $-$ & $2i)$ \\ 
% % &  &  &  &  &  &  &  &  &  &  \\ 
% % &  &  &  &  &  &  &  &  &  & 0 \\ 
% % &  &  &  &  &  &  &  &  &  & 
% % \end{tabular}
% % \end{equation*}
% \begin{eqnarray*}
% p(z) &=& (z-i)(z^2 - 3z + 2)\\
% \onslide<+->
% z^2 - 3z + 2 &=& 0 \onslide<+-> \Leftrightarrow z= 1 \text{ eller } z = 2
% \end{eqnarray*}
% \onslide<+->
% Ekvationen har alltså rötterna $z_{1}=i$, $z_{2}=1$ och $z_{3}=2$.
% \end{losning}
% \end{exempel}
% \end{frame}

% %-------------------------------------------------------------------------------
% \begin{frame}
% \frametitle{Polynom och algebraiska ekvationer}
% \begin{exempel}
% Lös ekvationen $p(z)=z^3 - 2z^2 + z =  0$.
% \onslide<+->
% \begin{losning}
% \begin{eqnarray*}
% p(z)&=&z^3 - 2z^2 + z = \onslide<+->z(z^2 - 2z + 1) =\onslide<+-> z(z - 1)^2 = 0 \\
% \onslide<+->
% \Leftrightarrow
% z&=&0 \text{ eller } z = 1.
% \end{eqnarray*}
% \end{losning}
% \end{exempel}
% \onslide<+->
% $(z-1)^{2}$ är en faktor i $p(z)$  \\
% \onslide<+->
% $\Rightarrow z=1$ är ett nollställe med multiplicitet 2 eller en dubbelrot till $p(z)=0$
% \begin{definition}
% Om $p(z)$ är ett polynom och $(z - \alpha)^k$ är en faktor i $p(z)$
% men inte $(z - \alpha)^{k + 1}$, $k \geq 1$, så har $p(z)$ nollstället $\alpha$ av 
% \alert{multiplicitet} $k$.
% \end{definition}
% \end{frame}

% \subsection{Algebrans fundamentalsats}
% %-------------------------------------------------------------------------------
% \begin{frame}
% \frametitle{Algebrans fundamentalsats}
% Hur många nollställen har ett polynom av grad $n$?
% \onslide<+->
% % Vi skall nu undersöka hur många nollställen ett polynom av grad $%
% % n $ har samt metoder för att bestämma dessa nollställen. Den
% % första fråga man kan ställa är om ett godtyckligt polynom av
% % grad $n\geq 1$ alltid har åtminstone ett nollställe. Svaret på 
% % den frågan är ja och gavs 1799 av den tyske
% % matematikern Gauss i sin doktorsavhandling. Den sats som garanterar detta
% % brukar kallas

% \begin{sats}[Algebrans fundamentalsats]
% \label{fundsats}
% Om $p(z)$ är ett polynom av grad $\geq 1$ finns det
% ett $\alpha \in \mathbb{C}$ sådant att $p(\alpha )=0$.
% \end{sats}
% \onslide<+->
% Faktorsatsen kombinerad med och algebrans fundamentalsats ger:

% \begin{sats}
% Varje polynom $p(z)$ av grad $n \geq 1$ har exakt $n$ nollställen i $\mathbb{C}$ om varje nollställe räknas med sin multiplicitet.
% \end{sats}
% \onslide<+->
% \begin{anm}
% \begin{columns}[c]
% \column{0.67\textwidth}
% Algebrans fundamentalsats bevisades av 1799 av den tyske
% matematikern Johann Carl Friedrich Gauss (1777-1855) i sin
% doktorsavhandling. Gauss anses vara en av de största matematikerna
% genom tiderna.
% \column{0.1\textwidth}
% \shadowimage[width=1.7cm]{bilder/Carl_Friedrich_Gauss.jpg}
% %\includegraphics[width=1.5cm]{bilder/Carl_Friedrich_Gauss.jpg}  
% \end{columns}
% \end{anm}
% \end{frame}

% %-------------------------------------------------------------------------------
% \begin{frame}
% \frametitle{Algebrans fundamentalsats}
% % Beviset för denna sats är ett enkelt induktionsbevis. 

% % \begin{exempel}
% % Bevisa Sats 1.
% % \end{exempel}

% \begin{itemize}
% \item Varje algebraisk ekvation av grad $n$ har
%  exakt $n$ rötter. 
% \item Om $p(z) = a_nz^n +...+ a_1z + a_0,\, (a_n\neq 0,\,n\geq 1)$ har
%   nollstället $z_1$ med multiplicitet $k_1$, $z_2$ med
%   multiplicitet $k_{2},...$ så är 
% \begin{eqnarray*}
% \onslide<+->
% \begin{cases}
% k_{1}+k_{2}+...+k_{m}=n\\
% \onslide<+->
% p(z)=a_{n}(z-z_{1})^{k_{1}}(z-z_{2})^{k_{2}}\cdots (z-z_{m})^{k_{m}}
% \end{cases}
% \end{eqnarray*}
% \end{itemize}
% \begin{exempel}
% Ange ett polynom av minsta möjliga grad som har $z=0$ som
% nollställe av multiplicitet 3, $z=1$ som dubbelt nollställe och $z=i$ som enkelt
% nollställe.
% \onslide<+->
% \begin{losning}
% Vi kan t ex ta polynomet 
% $p(z)=\onslide<+->
% z^{3}\onslide<+->
% (z-1)^{2}\onslide<+->
% (z-i)=\onslide<+->
% z^{6}-(2+i)z^{5}+(1+2i)z^{4}-iz^{3}$
% \end{losning}
% \end{exempel}
% \end{frame}

% %-------------------------------------------------------------------------------
% \begin{frame}
% \frametitle{Samband mellan nollställen och koefficienter}
% % \section{Samband mellan nollställen och koefficienter}

% Faktorsatsen ger samband mellan nollställen och koefficienter: 

% \begin{exempel}
% Om $p(z) = a_2z^2 + a_1z + a_0$ har nollställena $\alpha_1$ och
% $\alpha_2$ så är: 
% \onslide<+->
% \begin{eqnarray*}
% p(z) &=& a_2z^2 + a_1z + a_0 = \onslide<+->
% a_2(z - \alpha_1)(z - \alpha_2)
% \\
% \onslide<+->
% &=&a_2z^2 - a_2(\alpha_1 + \alpha_2)z + a_2\alpha_1\alpha_2\\ 
% \onslide<+->
% &\Leftrightarrow&
% \left\{ 
% \begin{array}{ll}
% \alpha_1+\alpha _{2} & =\displaystyle-\frac{a_{1}}{a_{2}} \\ 
% \onslide<+->
% \alpha_1\cdot \alpha _{2} & =\displaystyle\frac{a_{0}}{a_{2}}
% \end{array}
% \right.
% \end{eqnarray*}
% \end{exempel}

% \end{frame}

% %-------------------------------------------------------------------------------
% \begin{frame}
% \frametitle{Samband mellan nollställen och koefficienter}
% % \section{Samband mellan nollställen och koefficienter}
% \begin{itemize}
%  \item För ett polynom av grad $n$ gäller motsvarande samband
% \item Koefficientidentifieringen  för 2:a-gradspolynomet gav två samband
% \item För ett $n$:te grads polynom får vi $n$ samband
% \item För $n=3$, 
% \[
% p(z)=a_{3}z^{3}+a_{2}z^{2}+a_{1}z+a_{0}
% \]
% och nollställena $\alpha _{1}$, $\alpha _{2}$ och $\alpha _{3}$ får vi
% \onslide<+->
%   \begin{eqnarray*}
%     p(z) &=&a_{3}(z-\alpha _{1})(z-\alpha _{2})(z-\alpha _{3})=...  \notag \\
% \onslide<+->
%          &=&a_{3}z^{3}-a_{3}(\alpha _{1}+\alpha _{2}+\alpha _{3})z^{2}+a_{3}(\alpha
%     _{1}\alpha _{2}+\alpha _{2}\alpha _{3}+\alpha _{1}\alpha _{3})z-a_{3}\alpha
%     _{1}\alpha _{2}\alpha _{3}\   \\
% %  \end{eqnarray}
% %  sambandet mellan koefficienterna och nollställena till $p(z)$
% %  ges av
% %  \begin{equation*}
% \onslide<+->
%     &\Rightarrow& \left\{ 
%       \begin{array}{lll}
%         \alpha _{1}+\alpha _{2}+\alpha _{3} & =\displaystyle-\frac{a_{2}}{a_{3}} & 
%         \\ 
% \onslide<+->
%         \alpha _{1}\alpha _{2}+\alpha _{2}\alpha _{3}+\alpha _{1}\alpha _{3} & =%
%         \displaystyle\frac{a_{1}}{a_{3}} &  \\ 
% \onslide<+->
%         \alpha _{1}\alpha _{2}\alpha _{3} & =\displaystyle-\frac{a_{0}}{a_{3}} & 
%       \end{array}
%     \right.
% \end{eqnarray*}
% \end{itemize}

% \end{frame}

% %-------------------------------------------------------------------------------
% \begin{frame}
% \frametitle{Samband mellan nollställen och koefficienter}
% % % \section{Samband mellan nollställen och koefficienter}
% % I det allmänna fallet är två av sambanden är
% % speciellt intressanta nämligen rötternas summa och nollställenas
% % produkt: 
% % \begin{equation*}
% % \left\{ 
% % \begin{array}{ll}
% % \alpha _{1}+\alpha _{2}+...+\alpha _{n} & =\displaystyle-\frac{a_{n-1}}{a_{n}%
% % } \\ 
% % \alpha _{1}\alpha _{2}\cdots \alpha _{n} & =\displaystyle(-1)^{n}\frac{a_{0}%
% % }{a_{n}}
% % \end{array}
% % \right.
% % \end{equation*}

% \begin{exempel}
% Ekvationen $z^{3}-kz^{2}+kz-2=0$ har rötterna $\alpha _{1}$, $\alpha
% _{2} $ och $\alpha _{2}$. Visa att 
% \[
% \alpha _{1}(1-\alpha _{2})+\alpha
% _{2}(1-\alpha _{3})+\alpha _{3}(1-\alpha _{1})=0
% \]
% \onslide<+->
% \vspace{-0.3cm}
% \begin{losning}
% \begin{eqnarray*}
% \alpha_1(1 - \alpha_2) &+& \alpha_2(1 - \alpha_3) + \alpha_3(1
% -\alpha_1) \\
% \onslide<3->
% &=& \alpha_1 + \alpha_2 + \alpha_3 - (\alpha_1\alpha_2 +
% \alpha_2\alpha_3 + \alpha_1\alpha_3) \\
% \onslide<4->
% &=& - \frac{a_2}{a_3} - \frac{a_1}{a_3} = -(-k)-k = 0
% \end{eqnarray*}
% \end{losning}
% \end{exempel}
% \onslide<5->
% I det allmänna fallet får vi följande samband för rötternas
% summa och produkt
% \begin{equation*}
% \left\{ 
% \begin{array}{ll}
% \alpha _{1}+\alpha _{2}+...+\alpha _{n} & =\displaystyle-\frac{a_{n-1}}{a_{n}%
% } \\ 
% \alpha _{1}\alpha _{2}\cdots \alpha _{n} & =\displaystyle(-1)^{n}\frac{a_{0}%
% }{a_{n}}
% \end{array}
% \right.
% \end{equation*}
% \end{frame}

% \subsection{Algebraiska ekvationer med reella koefficienter}
% %-------------------------------------------------------------------------------
% \begin{frame}
% \frametitle{Algebraiska ekvationer med reella koefficienter}
% % \section{Algebraiska ekvationer med reella koefficienter}

% % Vi skal nu studera de algebraiska ekvationer som har reella koefficienter.
% % Dessa ekvationer visar sig ha den intressanta egenskapen att rötterna
% % alltid förekommer i par. \newline

% \begin{exempel}
% Lös ekvationen $z^{2}-2z+5=0$.
% \onslide<+->
% \begin{losning}
% \begin{eqnarray}
% z^{2}-2z+5 &=&\onslide<+->\left( z-1\right)^{2}-1+5=0  \notag \\
% \onslide<+->
% &\Leftrightarrow &\left( z-1\right) ^{2}=-4  \notag \\
% \onslide<+->
% &\Leftrightarrow &z-1=\pm 2i  \notag
% \end{eqnarray}
% \onslide<+->
% $\therefore z_0=1+2i$ och $\bar{z}_0=1-2i$ dvs rötterna är varandras konjugat.
% \end{losning}
% \end{exempel}
% \end{frame}
% %-------------------------------------------------------------------------------
% \begin{frame}
% \frametitle{Algebraiska ekvationer med reella koefficienter}
% Gäller detta allmänt?\\
% \onslide<+->
% \vspace{0.5cm}

% Antag att $p(z)=a_{n}z^{n}+...+a_{1}z+a_{0}$ har nollstället $z_0$:
% \begin{eqnarray*}
% \onslide<+->p(z_0)&=&a_nz_0^n + ... + a_1z_0 + a_0 = 0\\
% \onslide<+->
% \Rightarrow 0=\overline{0}=\overline{p(z_{0})} &=&\overline{a_{n}z_{0}^{n}+...+a_{1}z_{0}+a_{0}}  \notag \\
% \onslide<+->
% &=&\overline{a_{n}z_{0}^{n}}+...+\overline{a_{1}z_{0}}+\bar{a}_{0}\quad \leftarrow \text{konjugeringsreglerna}  \\
% \onslide<+->
% &=&a_{n}\bar{z}_{0}^{n}+...+a_{1}\bar{z}_{0}+a_{0}\quad \leftarrow  \text{alla koeff. är reella}\\
% \onslide<+->
% &=& p(\bar{z}_{0})
% \end{eqnarray*}
% \begin{sats}
% \label{relkoef}
% Om $p(z)$ är ett polynom med reella koefficienter och om $p(z)$ har
% nollstället $z_0 = a + ib, \: a, b \in \mathbb{R}, \: b \neq 0$ så 
% har $p(z)$ även nollstället $\bar{z}_0 = a - ib$.
% \end{sats}

% % \begin{Bevis}
% % Vi har $p(z)=a_{n}z^{n}+...+a_{1}z+a_{0}$ där $a_{0},a_{1},...,a_{n}\in $
% % \textbf{R} och $z_{0}$ är ett icke reellt nollställe 
% % \begin{equation*}
% % \Rightarrow p(z_{0})=a_{n}z_{0}^{n}+...+a_{1}z_{0}+a_{0}=0:::(\ast )
% % \end{equation*}
% % Komplexkonjugera $(\ast )$: 
% % \begin{eqnarray}
% % \Rightarrow \overline{p(z_{0})} &=&\overline{%
% % a_{n}z_{0}^{n}+...+a_{1}z_{0}+a_{0}}  \notag \\
% % &=&\overline{a_{n}z_{0}^{n}}+...+\overline{a_{1}z_{0}}+\bar{a}%
% % _{0}::\gets \mbox{konjugeringsreglerna}  \notag \\
% % &=&a_{n}\bar{z}_{0}^{n}+...+a_{1}\bar{z}_{0}+a_{0}=p(\bar{z}_{0})=\bar{0}%
% % =0::\gets \mbox{alla koeff. är reella}  \notag
% % \end{eqnarray}
% % Om $z_{0}$ är ett nollställe så är alltså också $\bar{z}_{0}$ ett nollställe.$%
% % \hfill $
% % \end{Bevis} 
% \onslide<8-> 
% \begin{anm}
% Enligt Sats \ref{relkoef} har nollställena $z_0$ och $\bar{z}_0$ samma multiplicitet.
% \end{anm}
% \end{frame}

% %-------------------------------------------------------------------------------
% \begin{frame}
% \frametitle{Algebraiska ekvationer med reella koefficienter}
% \begin{exempel}
% Polynomet $p(z)=z^{4}+2z^{3}-2z^{2}-8z-8$ har ett nollställe $z_{0}=-1+i$. Lös ekvationen $p(z)=0$.
% \onslide<+->
% \begin{losning}
% $p(z)$ har reella koefficienter  \onslide<+-> $\Rightarrow \bar{z}_{0}=-1-i$ är också ett nollställe enligt Sats \ref{relkoef}. \\
% \onslide<+->
% Faktorsatsen $\Rightarrow$ 
% \begin{eqnarray*}
% (z-z_{0})(z-\bar{z}_{0})&=&\onslide<+->z^2-z\bar{z}_0 -z_0z + z_0\bar{z}_0 =\onslide<+->
% z^2-(z_0+\bar{z}_0)z + z_0\bar{z}_0 \\
% &=& \onslide<+->z^2 - 2\text{Re} (z_0)z + |z_0|^2 = \onslide<+->z^2 + 2z + 2
% \end{eqnarray*}
% är en faktor i $p(z)$. \\
% \vspace{0.3cm}
% \onslide<+->
% Polynomdivision 
% % \begin{equation*}
% % \begin{tabular}{ccccccccccccccc}
% % &  &  &  &  &  & $z^{2}$ & $-$ & $4$ &  &  &  &  &  & $z^{2}$ \\ 
% % &  &  &  &  &  &  &  &  &  &  &  &  &  &  \\ 
% % $z^{2}$ & $+$ & $2z$ & $+$ & $2$ &  & $z^{4}$ & $+$ & $2z^{3}$ & $-$ & $%
% % 2z^{2}$ & $-$ & $8z$ & $-$ & $8$ \\ 
% % &  &  &  &  &  &  &  &  &  &  &  &  &  &  \\ 
% % &  &  &  &  & $-$ & $(z^{4}$ & $+$ & $2z^{3}$ & $+$ & $2z^{2})$ &  &  &  & 
% % \\ 
% % &  &  &  &  &  &  &  &  &  &  &  &  &  &  \\ 
% % &  &  &  &  &  &  &  &  &  & $-4z^{2}$ & $-$ & $8z$ & $-$ & $8$ \\ 
% % &  &  &  &  &  &  &  &  &  &  &  &  &  &  \\ 
% % &  &  &  &  &  &  &  &  & $-$ & $(-4z^{2}$ & $-$ & $8z$ & $-$ & $8)$ \\ 
% % &  &  &  &  &  &  &  &  &  &  &  &  &  &  \\ 
% % &  &  &  &  &  &  &  &  &  &  &  &  &  & $0$ \\ 
% % &  &  &  &  &  &  &  &  &  &  &  &  &  & 
% % \end{tabular}
% % \end{equation*}
% \onslide<+->
% $\Rightarrow p(z)=(z^{2}+2z+2)(z^{2}-4)$. 
% \onslide<+->
% \\ $p(z)=0$ har alltså rötterna $z=-1\pm i$ och $z=\pm 2$.
% \end{losning}
% \end{exempel}
% \end{frame}

% %-------------------------------------------------------------------------------
% \begin{frame}
% \frametitle{Algebraiska ekvationer med reella koefficienter}
% \begin{exempel}
% Bestäm konstanten $a$ så att polynomet 
% \begin{eqnarray*}
% p(z) &=&z^{4}-3z^{3}+z^{2}+az-2
% %\left( z-1\right) ^{2}\left( z+1\right) \left( z-2\right)
% %&=&z^{4}-3z^{3}+z^{2}+3z-2
% \end{eqnarray*}
% får faktorn $z^{2}-2z+1.$ Lös därefter ekvationen $p(z)=0$
% fullständigt.
% \onslide<+->
% \begin{losning}
% $z^{2}-2z+1=\left( z-1\right)^{2}$ är en faktor $p(z)$ omm $z=1$ är ett nollställe till $p(z)$
% \onslide<+->
% \begin{equation*}
% p(1)=\onslide<+->1-3+1+a-2=\onslide<+->a-3=\onslide<+->0\onslide<+->\Leftrightarrow a=3.
% \end{equation*}
% \onslide<+->
% Polynomdivision $\Rightarrow p(z)=(z^{2}-2z+1)( z^{2}-z-2)$ 
% \onslide<+->
% \begin{eqnarray*}
% p(z)=0 \Leftrightarrow z_1 = 1 
% \text{ eller }
%  z^{2}-z-2 = 0 
% \onslide<+->
% \Leftrightarrow
% z_2=-1, \, z_3=2
% \end{eqnarray*}
% \end{losning}
% \end{exempel}
% \end{frame}

% %-------------------------------------------------------------------------------
% \begin{frame}
% \frametitle{Algebraiska ekvationer med reella koefficienter}
% \begin{exempel}[Tenta 220104, uppgift 4b, 3p]
% Ekvationen 
% \[
% p(z) = z^5 -z^4 - z^3 + z^2 - 2z + 2  = 0
% \]
% har roten $i$. Lös ekvationen fullständigt och ange samtliga rötter på rektangulär form.
% \end{exempel}
% \onslide<+->
% \begin{losning}
% \vspace{-0.4cm}
% \begin{enumerate}
% \item $p(z)$ har reella koefficienter \onslide<+-> $\Rightarrow$ $-i$ är också en rot.
% \item Gissning ger roten $z=1$.
% \end{enumerate}
%   \onslide<+->
% 1\&2 $\Leftrightarrow p(z)$ har faktorn 
% \[
% (z-i)(z+i)(z-1) = (z^2 + 1)(z-1)=z^3-z^2+z-1
% \]
% \onslide<+->
% Polynomdivision $\Rightarrow p(z) =(z^3-z^2+z-1)(z^2-2)$
% \onslide<+->
% \[
% z^2-2 = 0 \Leftrightarrow z=\pm \sqrt{2}.
% \]
% \onslide<+->
% \emph{Svar:} $z = \pm i,\, z = 1\, $ och $z = \pm\sqrt{2}$.  
% \end{losning}
% \end{frame}
\end{document}


%%% Local Variables:
%%% mode: latex
%%% TeX-master: t
%%% End:
